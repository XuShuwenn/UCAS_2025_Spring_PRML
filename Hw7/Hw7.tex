\documentclass{article}
\usepackage{xeCJK} 
\usepackage{amsmath}
\usepackage{amsfonts}
\usepackage{amssymb}
\usepackage{graphicx} 
\usepackage{tikz}
\usepackage{CJKutf8}
\usepackage{pgfplots}
\pgfplotsset{compat=1.18}


\setCJKmainfont{SimSun}
\renewcommand\CJKfamilydefault{\CJKrmdefault}
\usepackage{sectsty}
\sectionfont{\normalfont}

\usetikzlibrary{shapes.geometric, arrows}

\tikzstyle{startstop} = [rectangle, rounded corners, minimum width=3cm, minimum height=1cm, text centered, draw=black, fill=red!30]
\tikzstyle{process} = [rectangle, minimum width=3cm, minimum height=1cm, text centered, draw=black, fill=blue!20]
\tikzstyle{decision} = [diamond, minimum width=3.5cm, minimum height=1cm, text centered, draw=black, fill=green!20]
\tikzstyle{arrow} = [thick,->,>=stealth]

\begin{document}
\title{PRML第五次作业}
\author{许书闻 2023K8009926005}
\maketitle
\section*{第1题}
(a)
\[P=(1-\frac{1}{N})^N\]
(b)
\[P=(1-\frac{1}{N})^{NB}\]
(c)

当B=N时,样本一次都未被选中的概率:

\[P=\lim_{N \to \infty}(1-\frac{1}{N})^{N^2}=\lim_{N\to \infty}[(1-\frac{1}{N})^{-N}]^{-N}=\lim_{N\to\infty}\frac{1}{e^N}=0  \]  
所以至少被选中一次的概率为\[1-P=1\]
(d)
\[200\times (1-\frac{1}{N})^N=200\times 0.368=74\]

\section*{第2题}

\subsection*{两种方法的区别}

\subsubsection*{1.1 Bagging 的机制}
\begin{itemize}
    \item \textbf{核心思想}:Bagging(Bootstrap Aggregating)通过从原始训练数据中有放回地抽取样本子集(Bootstrap 采样),为每棵决策树生成不同的训练集,从而增加模型的多样性。
    \item \textbf{特征选择}:在节点分裂时,Bagging 考虑\emph{所有特征},并选择最优分裂特征(基于信息增益或基尼指数等)。
    \item \textbf{多样性来源}:多样性仅来源于训练样本的随机性。由于每棵树的训练数据不同,树之间的结构和预测存在差异,但特征选择过程无额外限制。
\end{itemize}

\subsubsection*{1.2 随机森林的机制}
\begin{itemize}
    \item \textbf{核心思想}:随机森林在 Bagging 基础上进一步引入特征维度的随机性。每棵决策树在节点分裂时,从\emph{随机抽取的特征子集}中选择最优分裂特征(通常特征子集大小为 $\sqrt{n}$ 或 $\log_2(n)$,其中 $n$ 为总特征数)。
    \item \textbf{特征选择}:在每次节点分裂时,随机森林随机选择一个特征子集,然后在该子集内寻找最优分裂点。
    \item \textbf{多样性来源}:多样性来源于\emph{样本子集的随机性}(继承自 Bagging)和\emph{特征子集的随机性},使树之间的结构差异更大。
\end{itemize}

\subsubsection*{1.3 差异总结}
\begin{itemize}
    \item \textbf{样本随机性}:两者均使用 Bootstrap 采样生成样本子集,样本随机性无本质区别。
    \item \textbf{特征随机性}:随机森林额外引入特征子集的随机选择,而 Bagging 始终考虑所有特征。这增加了随机森林中树之间的独立性和多样性。
    \item \textbf{模型复杂度}:随机森林的特征子集限制使单棵树的“视野”更窄,预测能力可能较弱(偏差略高),但集成效果通常更好。
\end{itemize}

\subsection*{2. 对偏差和方差的影响}

集成模型的偏差和方差可分解为单棵树的偏差、单棵树的方差以及树之间的相关性。以下从理论角度分析。

\subsubsection*{2.1 Bagging 的偏差与方差}
\begin{itemize}
    \item \textbf{偏差}:
    \begin{itemize}
        \item Bagging 使用完整特征集构建每棵决策树,单棵树的偏差较低,接近于单棵完整决策树的偏差。
        \item 集成模型通过平均(分类为投票,回归为平均)多个树的预测,偏差通常与单棵树相当,因为平均不显著改变偏差。
    \end{itemize}
    \item \textbf{方差}:
    \begin{itemize}
        \item Bagging 的主要目标是降低方差。通过 Bootstrap 采样,每棵树基于不同训练子集,树之间的预测差异被平均化,显著降低集成模型的方差。
        \item 但由于每棵树考虑所有特征,树之间的相关性可能较高(尤其当某些特征特别重要时),限制了方差降低的幅度。
    \end{itemize}
\end{itemize}

\subsubsection*{2.2 随机森林的偏差与方差}
\begin{itemize}
    \item \textbf{偏差}:
    \begin{itemize}
        \item 随机森林因节点分裂时只考虑部分特征子集,单棵树的预测能力较 Bagging 弱,单棵树的偏差略高(因无法始终选择全局最优分裂特征)。
        \item 集成模型的偏差差异不大,因为大量树的集成能逼近整体数据模式,偏差接近单棵完整决策树的偏差。
    \end{itemize}
    \item \textbf{方差}:
    \begin{itemize}
        \item 随机森林通过特征子集的随机选择,降低树之间的相关性 $\rho$,显著降低集成模型的方差。
        \item 集成模型方差可表示为:
        \[
        \text{Var}(\hat{f}) = \rho \cdot \sigma^2 + \frac{1 - \rho}{T} \cdot \sigma^2
        \]
        其中 $\hat{f}$ 为集成模型预测,$\rho$ 为树间相关性,$\sigma^2$ 为单棵树预测方差,$T$ 为树的棵数。随机森林通过降低 $\rho$ 显著减少方差。
    \end{itemize}
\end{itemize}

\subsubsection*{2.3 偏差与方差比较}
\begin{itemize}
    \item \textbf{偏差}:随机森林的单棵树偏差略高于 Bagging,但集成模型偏差差异不大。
    \item \textbf{方差}:随机森林通过降低树间相关性,集成模型的方差通常低于 Bagging。
    \item \textbf{总体效果}:随机森林牺牲单棵树的预测能力(偏差略增),换取更大树间多样性,从而显著降低方差,通常在泛化性能上优于 Bagging。
\end{itemize}

\section*{第3题}
(a)
\[\epsilon_1=\sum_{i=1}^{5}0.2\times\mathcal{I}\{y_i\neq h_t(x_i)\}=0.2\times (1+1)=0.4\]
(b)
\[\alpha_1=\frac{1}{2}\log(\frac{1-0.4}{0.4})=0.088\]
(c)
\[\omega_{i}^{(2)}=\frac{\omega_i^{(1)}exp(-\alpha_ty_ih_t(x_i))}{Z_t},~~Z_t=0.2\times(3\cdot e^{-0.088}+2\cdot e^{0.088})=0.98628\]

所以更新后的权重向量为
\[\omega_1=\omega_4=\omega_5=\frac{0.5\times0.9158}{0.98628}=0.186,\omega_2=\omega_3=\frac{0.2\times1.092}{0.98628}=0.221\]

\section*{第4题}

\subsection*{step1. 欧氏距离计算}
对于样本 \( \mathbf{x}_i \) 和 \( \mathbf{x}_j \),欧氏距离为:
\[
d(\mathbf{x}_i, \mathbf{x}_j) = \sqrt{\sum_{k=1}^5 (x_{i,k} - x_{j,k})^2}
\]
计算所有样本对的距离,得到初始距离矩阵:
\[
\begin{array}{c|ccccc}
 & \mathbf{x}_1 & \mathbf{x}_2 & \mathbf{x}_3 & \mathbf{x}_4 & \mathbf{x}_5 \\
\hline
\mathbf{x}_1 & 0 & 1.732 & 3.873 & 2.449 & 3.317 \\
\mathbf{x}_2 & 1.732 & 0 & 2.449 & 2.236 & 2.828 \\
\mathbf{x}_3 & 3.873 & 2.449 & 0 & 3.606 & 2.449 \\
\mathbf{x}_4 & 2.449 & 2.236 & 3.606 & 0 & 2.646 \\
\mathbf{x}_5 & 3.317 & 2.828 & 2.449 & 2.646 & 0 \\
\end{array}
\]

\subsection*{step2. 层次化聚类过程}
使用\emph{单链接}(最近距离准则),簇间距离定义为:
\[
d(C_i, C_j) = \min_{\mathbf{x} \in C_i, \mathbf{y} \in C_j} d(\mathbf{x}, \mathbf{y})
\]
采用凝聚型层次聚类,从每个样本作为一个单独的簇开始,逐步合并距离最近的簇。

\subsubsection*{步骤 1:初始簇}
初始簇为:\( \{ \mathbf{x}_1 \}, \{ \mathbf{x}_2 \}, \{ \mathbf{x}_3 \}, \{ \mathbf{x}_4 \}, \{ \mathbf{x}_5 \} \).

\subsubsection*{步骤 2:第一次合并}
\begin{itemize}
    \item 最小距离:\( d(\mathbf{x}_1, \mathbf{x}_2) = 1.732 \).
    \item 合并 \( \mathbf{x}_1 \) 和 \( \mathbf{x}_2 \) 形成簇 \( C_1 = \{ \mathbf{x}_1, \mathbf{x}_2 \} \),高度为 1.732。
    \item 更新距离矩阵:
    \[
    \begin{array}{c|cccc}
     & C_1 & \mathbf{x}_3 & \mathbf{x}_4 & \mathbf{x}_5 \\
    \hline
    C_1 & 0 & 2.449 & 2.236 & 2.828 \\
    \mathbf{x}_3 & 2.449 & 0 & 3.606 & 2.449 \\
    \mathbf{x}_4 & 2.236 & 3.606 & 0 & 2.646 \\
    \mathbf{x}_5 & 2.828 & 2.449 & 2.646 & 0 \\
    \end{array}
    \]
\end{itemize}

\subsubsection*{步骤 3:第二次合并}
\begin{itemize}
    \item 最小距离:\( d(C_1, \mathbf{x}_4) = 2.236 \).
    \item 合并 \( C_1 \) 和 \( \mathbf{x}_4 \) 形成簇 \( C_2 = \{ \mathbf{x}_1, \mathbf{x}_2, \mathbf{x}_4 \} \),高度为 2.236。
    \item 更新距离矩阵:
    \[
    \begin{array}{c|ccc}
     & C_2 & \mathbf{x}_3 & \mathbf{x}_5 \\
    \hline
    C_2 & 0 & 2.449 & 2.646 \\
    \mathbf{x}_3 & 2.449 & 0 & 2.449 \\
    \mathbf{x}_5 & 2.646 & 2.449 & 0 \\
    \end{array}
    \]
\end{itemize}

\subsubsection*{步骤 4:第三次合并}
\begin{itemize}
    \item 最小距离:\( d(C_2, \mathbf{x}_3) = 2.449 \)(与 \( d(\mathbf{x}_3, \mathbf{x}_5) = 2.449 \) 相等,选择前者)。
    \item 合并 \( C_2 \) 和 \( \mathbf{x}_3 \) 形成簇 \( C_3 = \{ \mathbf{x}_1, \mathbf{x}_2, \mathbf{x}_3, \mathbf{x}_4 \} \),高度为 2.449。
    \item 更新距离矩阵:
    \[
    \begin{array}{c|cc}
     & C_3 & \mathbf{x}_5 \\
    \hline
    C_3 & 0 & 2.449 \\
    \mathbf{x}_5 & 2.449 & 0 \\
    \end{array}
    \]
\end{itemize}

\subsubsection*{步骤 5:第四次合并}
\begin{itemize}
    \item 最小距离:\( d(C_3, \mathbf{x}_5) = 2.449 \).
    \item 合并 \( C_3 \) 和 \( \mathbf{x}_5 \) 形成最终簇 \( C_4 = \{ \mathbf{x}_1, \mathbf{x}_2, \mathbf{x}_3, \mathbf{x}_4, \mathbf{x}_5 \} \),高度为 2.449。
\end{itemize}

\subsection*{4. 聚类分级树}
聚类过程可以用树状图表示,高度对应合并时的距离:
\begin{itemize}
    \item 高度 1.732:合并 \( \mathbf{x}_1 \) 和 \( \mathbf{x}_2 \) 形成 \( C_1 \).
    \item 高度 2.236:合并 \( C_1 \) 和 \( \mathbf{x}_4 \) 形成 \( C_2 \).
    \item 高度 2.449:合并 \( C_2 \) 和 \( \mathbf{x}_3 \) 形成 \( C_3 \).
    \item 高度 2.449:合并 \( C_3 \) 和 \( \mathbf{x}_5 \) 形成 \( C_4 \).
\end{itemize}

以下是聚类分级树:
\begin{center}
\begin{tikzpicture}
    % 节点(样本)
    \node (x1) at (0,0) {\( \mathbf{x}_1 \)};
    \node (x2) at (2,0) {\( \mathbf{x}_2 \)};
    \node (x4) at (4,0) {\( \mathbf{x}_4 \)};
    \node (x3) at (6,0) {\( \mathbf{x}_3 \)};
    \node (x5) at (8,0) {\( \mathbf{x}_5 \)};

    % 合并节点
    \node (c1) at (1,1.732) {};
    \node (c2) at (2,2.236) {};
    \node (c3) at (4,2.449) {};
    \node (c4) at (4.5,2.449) {};

    % 连接线
    \draw (x1) -- (c1) node[midway, left] {1.732};
    \draw (x2) -- (c1);
    \draw (c1) -- (c2) node[midway, left] {2.236};
    \draw (x4) -- (c2);
    \draw (c2) -- (c3) node[midway, left] {2.449};
    \draw (x3) -- (c3);
    \draw (c3) -- (c4) node[midway, above] {2.449};
    \draw (x5) -- (c4);

    % 高度轴
    \draw[->] (-1,0) -- (-1,3) node[left] {高度};
    \foreach \y/\label in {1.732/1.732, 2.236/2.236, 2.449/2.449}
        \draw (-1.1,\y) -- (-0.9,\y) node[left] {\label};
\end{tikzpicture}
\end{center}

\section*{第5题}

\subsection*{(a)初始中心 \( \mathbf{x}_1 = (0, 0) \), \( \mathbf{x}_5 = (6, 6) \)}
\subsubsection*{第1轮迭代}
初始中心:
\begin{itemize}
    \item \( C_1 = (0, 0) \)
    \item \( C_2 = (6, 6) \)
\end{itemize}
\textbf{分配样本点}:计算每个样本到 \( C_1 \), \( C_2 \) 的距离,分配到最近中心。结果:
\begin{itemize}
    \item 簇 1:\( \{ \mathbf{x}_1, \mathbf{x}_2, \mathbf{x}_3, \mathbf{x}_4 \} \)
    \item 簇 2:\( \{ \mathbf{x}_5, \mathbf{x}_6, \mathbf{x}_7, \mathbf{x}_8 \} \)
\end{itemize}
\textbf{更新中心}:
\begin{itemize}
    \item 簇 1:\( C_1 = \left( \frac{0+2+0+2}{4}, \frac{0+0+2+2}{4} \right) = (1, 1) \)
    \item 簇 2:\( C_2 = \left( \frac{6+8+6+8}{4}, \frac{6+6+8+8}{4} \right) = (7, 7) \)
\end{itemize}

\subsubsection*{第2轮迭代}
新中心:
\begin{itemize}
    \item \( C_1 = (1, 1) \)
    \item \( C_2 = (7, 7) \)
\end{itemize}
\textbf{分配样本点}:分配结果与迭代 1 相同:
\begin{itemize}
    \item 簇 1:\( \{ \mathbf{x}_1, \mathbf{x}_2, \mathbf{x}_3, \mathbf{x}_4 \} \)
    \item 簇 2:\( \{ \mathbf{x}_5, \mathbf{x}_6, \mathbf{x}_7, \mathbf{x}_8 \} \)
\end{itemize}
\textbf{更新中心}:中心未变化,算法收敛。

\textbf{最终结果}:
\begin{itemize}
    \item 簇 1:\( \{ \mathbf{x}_1, \mathbf{x}_2, \mathbf{x}_3, \mathbf{x}_4 \} \),中心 \( (1, 1) \).
    \item 簇 2:\( \{ \mathbf{x}_5, \mathbf{x}_6, \mathbf{x}_7, \mathbf{x}_8 \} \),中心 \( (7, 7) \).
\end{itemize}

\subsection*{(b)初始中心 \( \mathbf{x}_2 = (2, 0) \), \( \mathbf{x}_4 = (2, 2) \)}
\subsubsection*{第1轮迭代}
初始中心:
\begin{itemize}
    \item \( C_1 = (2, 0) \)
    \item \( C_2 = (2, 2) \)
\end{itemize}
\textbf{分配样本点}:
\begin{itemize}
    \item 簇 1:\( \{ \mathbf{x}_1, \mathbf{x}_2 \} \)
    \item 簇 2:\( \{ \mathbf{x}_3, \mathbf{x}_4, \mathbf{x}_5, \mathbf{x}_6, \mathbf{x}_7, \mathbf{x}_8 \} \)
\end{itemize}
\textbf{更新中心}:
\begin{itemize}
    \item 簇 1:\( C_1 = \left( \frac{0+2}{2}, \frac{0+0}{2} \right) = (1, 0) \)
    \item 簇 2:\( C_2 = \left( \frac{0+2+6+8+6+8}{6}, \frac{2+2+6+6+8+8}{6} \right) = (5, 5.333) \)
\end{itemize}

\subsubsection*{第2轮迭代}
新中心:
\begin{itemize}
    \item \( C_1 = (1, 0) \)
    \item \( C_2 = (5, 5.333) \)
\end{itemize}
\textbf{分配样本点}:
\begin{itemize}
    \item 簇 1:\( \{ \mathbf{x}_1, \mathbf{x}_2, \mathbf{x}_3, \mathbf{x}_4 \} \)
    \item 簇 2:\( \{ \mathbf{x}_5, \mathbf{x}_6, \mathbf{x}_7, \mathbf{x}_8 \} \)
\end{itemize}
\textbf{更新中心}:
\begin{itemize}
    \item 簇 1:\( C_1 = \left( \frac{0+2+0+2}{4}, \frac{0+0+2+2}{4} \right) = (1, 1) \)
    \item 簇 2:\( C_2 = \left( \frac{6+8+6+8}{4}, \frac{6+6+8+8}{4} \right) = (7, 7) \)
\end{itemize}

\subsubsection*{第3轮迭代}
新中心:
\begin{itemize}
    \item \( C_1 = (1, 1) \)
    \item \( C_2 = (7, 7) \)
\end{itemize}
\textbf{分配样本点}:分配结果与迭代 2 相同,中心未变化,算法收敛。

\textbf{最终结果}:
\begin{itemize}
    \item 簇 1:\( \{ \mathbf{x}_1, \mathbf{x}_2, \mathbf{x}_3, \mathbf{x}_4 \} \),中心 \( (1, 1) \).
    \item 簇 2:\( \{ \mathbf{x}_5, \mathbf{x}_6, \mathbf{x}_7, \mathbf{x}_8 \} \),中心 \( (7, 7) \).
\end{itemize}

\subsection*{(c)初始点选取对聚类效果的影响}

\begin{itemize}
    \item \( \mathbf{x}_1, \mathbf{x}_2, \mathbf{x}_3, \mathbf{x}_4 \) 靠近原点,形成紧凑簇。
    \item \( \mathbf{x}_5, \mathbf{x}_6, \mathbf{x}_7, \mathbf{x}_8 \) 位于右上区域,形成另一紧凑簇。
\end{itemize}

\textbf{分析}:
\begin{itemize}
    \item \textbf{情况 (a)}:初始中心 \( \mathbf{x}_1 = (0, 0) \), \( \mathbf{x}_5 = (6, 6) \) 分别位于两个自然簇,接近真实簇中心。算法在 2 次迭代后收敛,簇分配与数据分布一致,簇内方差小,聚类效果优异。
    \item \textbf{情况 (b)}:初始中心 \( \mathbf{x}_2 = (2, 0) \), \( \mathbf{x}_4 = (2, 2) \) 均位于第一个自然簇,未能代表第二个簇。第一次迭代分配不理想,需 3 次迭代收敛到与情况 a 相同的簇划分,效率较低。
    \item \textbf{初始点影响}:
    \begin{itemize}
        \item \emph{初始点位置}:情况 a 的初始中心分布合理,快速收敛。情况 b 的初始中心过于集中,初期分配偏离真实结构。
        \item \emph{收敛速度}:合理初始点(情况 a)减少迭代次数。初始点不当(情况 b)增加迭代次数,但最终结果一致。
        \item \emph{局部最优风险}:本例数据分布明显,两种初始点均收敛到全局最优。但在复杂数据中,不当初始点可能导致局部最优。
    \end{itemize}
\end{itemize}


\end{document}
